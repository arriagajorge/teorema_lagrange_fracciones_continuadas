\documentclass[11pt, article]{article}
\pagestyle{empty}
% pre\'ambulo

\usepackage{lmodern}
\usepackage[T1]{fontenc}
%\usepackage[spanish,activeacute]{babel}
\usepackage[spanish,es-tabla]{babel}
\usepackage{mathtools}
\usepackage{babel}
\usepackage{textcomp}
\usepackage[utf8]{inputenc}
\usepackage{vmargin}
\usepackage{amsfonts}
\usepackage{mathrsfs}
\usepackage{graphicx}
\usepackage{enumerate}
\usepackage{subcaption}
%\usepackage[utf8]{inputenc}

\setpapersize{A4}
\setmargins{1.5cm}             % margen izquierdo
{1.5cm}                        % margen superior
{18cm}                         % anchura del texto
{25.42cm}                      % altura del texto
{10pt}                         % altura de los encabezados
{0cm}                          % espacio entre el texto y los encabezados
{0pt}                          % altura del pie de página
{1cm}                          % espacio entre el texto y el pie de página

\title{Fracciones continuadas}
\author{}
\date{}
\renewcommand{\baselinestretch}{1.5}

\begin{document}
\maketitle
\thispagestyle{empty}
\section*{Introducción}
Las fracciones continuadas tienen su comienzo como...
\section*{Sobre el Teorema de Lagrange}
La motivación es demostrar el teorema de Lagrange

\textbf{Teorema de Lagrange} Sea $\alpha \in \mathbb{R}$. La fraccion continuada de $\alpha$ es periodica si y solo si existen $a, b, c \in\mathbb{Z}$ con $a\not=0$ tales que $a\alpha^2 + b\alpha + c = 0$.

Empecemos suponiendo que $\alpha$ es una fracción continuada periódica, de lo cual $\alpha= [a_0,a_1,...,a_{j,\theta}]$, donde $\theta=\overline{[b_0,b_1,..,b_{n-1}]}$, por la ecuación (7.8) se sigue que
\[
\theta =\dfrac{\theta h_{n-1}+h_{n-2}}{\theta k_{n-1}+k_{n-2}}
\]
y esta es una ecuación cuádratica en $\theta$, de aquí $\theta$ es un número irracional cuadrático o bien un número irracional pero el valor de cualquier fracción continuada simple infinita es irracional(dem), ahora $\alpha$ puede escribirse como 
\[
\alpha = [a_0,a_1,..,a_j,\theta]=\dfrac{\theta m + m'}{\theta q + q'}
\]
donde $\dfrac{m'}{q'}$ y $\dfrac{m}{q}$ son los dos últimos convergentes para $[a_0,a_1,...,a_j]$, pero $\theta$ es de la forma $\dfrac{(a + \sqrt{b})}{c}$ y de aquí $\alpha$ es de forma semejante, debido a que como con $\theta$ puede excluirse la posibilidad de que $\alpha$ sea racional

\end{document}