\documentclass[11pt, article]{article}
\pagestyle{empty}
% pre\'ambulo

\usepackage{lmodern}
\usepackage[T1]{fontenc}
%\usepackage[spanish,activeacute]{babel}
\usepackage[spanish,es-tabla]{babel}
\usepackage{mathtools}
\usepackage{babel}
\usepackage{textcomp}
\usepackage[utf8]{inputenc}
\usepackage{vmargin}
\usepackage{amsfonts}
\usepackage{mathrsfs}
\usepackage{graphicx}
\usepackage{enumerate}
\usepackage{subcaption}
%\usepackage[utf8]{inputenc}

\setpapersize{A4}
\setmargins{1.5cm}             % margen izquierdo
{1.5cm}                        % margen superior
{18cm}                         % anchura del texto
{25.42cm}                      % altura del texto
{10pt}                         % altura de los encabezados
{0cm}                          % espacio entre el texto y los encabezados
{0pt}                          % altura del pie de página
{1cm}                          % espacio entre el texto y el pie de página

\title{Fracciones continuadas}
\author{}
\date{}
\renewcommand{\baselinestretch}{1.5}

\begin{document}
\maketitle
\thispagestyle{empty}
\section*{Introducción}
    Las fracciones continuadas tienen su comienzo como...
\section*{Sobre el Teorema de Lagrange}
    La motivación es demostrar el teorema de Lagrange

    \textbf{Teorema de Lagrange} Sea $\alpha \in \mathbb{R}$. La fraccion continuada de $\alpha$ es periodica si y solo si existen $a, b, c \in\mathbb{Z}$ con $a\not=0$ tales que $a\alpha^2 + b\alpha + c = 0$.

    Empecemos suponiendo que $\alpha$ es una fracción continuada periódica, de lo cual $\alpha= [a_0,a_1,...,a_{j,\theta}]$, donde $\theta=\overline{[b_0,b_1,..,b_{n-1}]}$, por la ecuación (7.8) se sigue que
        \[
        \theta =\dfrac{\theta h_{n-1}+h_{n-2}}{\theta k_{n-1}+k_{n-2}}
        \]
    y esta es una ecuación cuadrática en $\theta$, de aquí $\theta$ es un número irracional cuadrático o bien un número irracional pero el valor de cualquier fracción continuada simple infinita es irracional(dem), ahora $\alpha$ puede escribirse como 
        \[
        \alpha = [a_0,a_1,..,a_j,\theta]=\dfrac{\theta m + m'}{\theta q + q'}
        \]
    donde $m'/q'$ y $m/q$ son los dos últimos convergentes para $[a_0,a_1,...,a_j]$, pero $\theta$ es de la forma $(a + \sqrt{b})/c$ y de aquí $\alpha$ es de forma semejante, debido a que como con $\theta$ puede excluirse la posibilidad de que $\alpha$ sea racional

    Ahora supongamos que $\alpha$ es irracional cuadrático, entonces es de la forma $\alpha=(a+\sqrt{b})/c$, con los enteros $a,b,c,d>0$, $c\not=0$, el entero $b$ no es un cuadrado perfecto puesto que $\alpha$ es irracional. Multiplequemos el numerador y el denominador por $|c|$ para obtener 
        \[
        \alpha = \dfrac{ac + \sqrt{bc^2}}{c^2} \textnormal{o bien } \alpha=\dfrac{-ac+\sqrt{bc^2}}{-x^2}
        \]
    de acuerdo con que $c$ sea positiva o negativa. Así puede escribirse $\alpha$ de la forma
        \[
        \alpha = \dfrac{m_0+\sqrt{d}}{q_0}
        \]
    donde $q_0|(d-m_0^2)$, $d,m_0$ y $q$ son enteros, $q_0\not=0$, $d$ no es un entero cuadrado perfecto. Escribiendo $\alpha$ en esta forma podemos obtener una formulación sencilla de su desarrlo fraccionario continuado $\overline{[a_0,a_1,...]}$ . Se probara que las siguientes ecuaciones 
        \[
        a_i=[\alpha_i], \qquad \alpha_i=\dfrac{m_i+\sqrt{d}}{q_i},
        \]
        \[
        m_{i+1}=a_iq_i-m_i, \qquad q_{i+1}=\dfrac{d-m_{i+1}^2}{q_i}
        \]
    definen las sucesiones infinitas de los enteros $m_i,q_i,a_i$ e irracionales $\alpha_i$ de tal forma que se cumplen las ecuaciones $(7.7)dem$ y de donde se obtendra el desarrollo continuado de $\alpha$

    En primer lugar, empecemos con $\alpha_0,m_0,q_0$ tal y cimo se determinaron anteriormente  y hagmaos $a_0=[\alpha_0]$. Si se conocen $\alpha_i,m_i,q_i,a_i$, entonces se tiene que $m_{i+1}=a_iq_i - m_i$, $q_i=(d-m_{i+q}^2)/q_i, \alpha_{i+1}=(m_{i+1} + \sqrt{d})/q_{i+1}, a_i=[\alpha_{i+1}]$. Es decir las ecuaciones anteriore $(poner numero)$ realmente determinan las sucesiones $\alpha_i,m_i,q_i,a_i$ que son por lo menos reales.
    
    Ahora se aplica inducción para probar que los $m_i$ y los $q_i$son enteros tales que $q_i\not=0$ y $q_i|(d-m^2)$. Esto se cumple para $i=0$. Supongamos que es cierto para el $i-$ésimo paso enttonces, se observa que $m_{i+1}=a_iq_i-m_i$ es n entero.Entonces la ecuación 
         \[
        q_{i+1}=\dfrac{d-m_{i+1}^2}{q_i}=\dfrac{d-m_i^2}{q_i} + 2a_im_i-a_i^2q_i
        \]
    establece que $q_{i+1}$ es un entero. Es más $q_i+1$ no puede ser cero, dado que si lo fuera, se tendría $d=m^2_{i+1}$, pero $d$ no es un cuadrado perfecto. Finalmente, se tiene que $q_i=(d-m_{i+1}^2)/q_{i+1}$, de modo que $q_{i+1}|(d-m_{i+1}^2)$
    
    A continuación puede verificarse que 
        \[
        \alpha_i-a_i=\dfrac{-a_iq_i+m_i+\sqrt{d}}{q_i}=\dfrac{\sqrt{d}-m_{i+1}}{q_i}=\dfrac{d-m_{i+1}^2}{q_i(\sqrt{d}+m_{i+1})}=\dfrac{q_i+1}{\sqrt{d}+m_{i+1}}=\dfrac{1}{\alpha_{i+1}}
        \]
    lo cual verifica $(7.7)dem$ y así se ha probado que $\alpha=[a_0,a_1,a_2,...]$, con los $a_i$ definidos por las ecuaciones $nombrarlas$ 
    
    Mediante $\alpha^{'}_{i}$ denotamos el conjunto de $\alpha_i$ esto es $\alpha^{'}_{i}=(m_i-\sqrt{d})/q_i$. Dado que el conjunto de un cociente es igual al cociente de lo conjufado se obtiene la ecuación
        \[
        \alpha_0{'}=\dfrac{\alpha_n^{'}h_{n-1}+h_{n-2}}{\alpha_n^{'}k_{n-1}+k_{n-2}}
        \]
    tomando los conjugado en $?(7.8)$y resolviendo para $\alpha_n^{'}$ se tiene
        \[
        \alpha_n^{'}=-\dfrac{k_{n-2}}{k_{n-1}}\left(\dfrac{\alpha_0^{'}-h_{n-2}/k_{n-2}}{\alpha_0^{'}-h_{n-1}/k_{n-1}}\right)
        \]
        
    Conforme $n$ tiende al infinito, tanto $h_{n-1}/k_{n-1}$ como $h_{m-2}/k_{n-2}$ tienden hacia $\alpha_0^{'}$ y por tanto la fracción que se encuentra dentro del paréntesis tiende a 1. Así que para un $n$ lo suficientemente grande, digamos $n>N$, 
    donde $N$ es fijo, la fracción del paréntesis es positiva y $\alpha_n^{'}$ es negativo, pero recordemos que $\alpha_n^{'}$ es positivo para toda
    $n\geq 1$ y de aqí que $\alpha_n - \alpha_n^{'}>0$ para $n>N$. Aplicando (7.16) se ve que da $2 \sqrt{d}/q_{n} >0$
    de aquí que $q_n>0$ para $n>N$.
    
    De (7.16) también se deduce que 
        \[
        q_nq_{n+1}=d-m^2_{n+1}\leq d, \qquad q_n\leq q_nq_{n+1}\leq d 
        \]
        \[
        m^2_{n+1} < m_{n+1}^2+q_n q_n+1=d, \qquad |m_{n+1}|<\sqrt{d}
        \]
    para $n>N$. Supuesto de que $d$ es un enttero positivo fijo se concluye que $q_n$ y $m_{n+1}$ pueden saumir un solo numero fijo de valores posibes para $q_n$ y $m_{n+1}$ pueden asumir sólo un número fijo de valores posibles para $n>N$. De aquñi que las parejas ordenadas $(m_n,q_n)$ pueden asumir sólo un número fijo de valores posibles de la parejas para $n>N$, y por tanto existen enteros distints $j$ y $k$ tales que $m_j=m_k$ y $q_j=q_k$, pueden suponerse que se han escogido $j$ y $k$ tales que $j<k$. Por (7.16), esto implica que $\alpha_j=\alpha_k$ y de aquí que
        \[
        \alpha=[a_0,a_1,...,a_{j-1},\overline{a_j,a_{j+1},...,a_{k-1}}]
        \]
\end{document}
