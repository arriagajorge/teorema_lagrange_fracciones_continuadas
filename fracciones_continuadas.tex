\documentclass[11pt, article]{article}
\pagestyle{empty}
% pre\'ambulo

\usepackage{lmodern}
\usepackage[T1]{fontenc}
%\usepackage[spanish,activeacute]{babel}
\usepackage[spanish,es-tabla]{babel}
\usepackage{mathtools}
\usepackage{babel}
\usepackage{textcomp}
\usepackage[utf8]{inputenc}
\usepackage{vmargin}
\usepackage{amsfonts}
\usepackage{mathrsfs}
\usepackage{graphicx}
\usepackage{enumerate}
\usepackage{subcaption}
%\usepackage[utf8]{inputenc}

\setpapersize{A4}
\setmargins{1.5cm}             % margen izquierdo
{1.5cm}                        % margen superior
{18cm}                         % anchura del texto
{25.42cm}                      % altura del texto
{10pt}                         % altura de los encabezados
{0cm}                          % espacio entre el texto y los encabezados
{0pt}                          % altura del pie de página
{1cm}                          % espacio entre el texto y el pie de página

\title{Fracciones continuadas}
\author{}
\date{}
\renewcommand{\baselinestretch}{1.5}

\begin{document}
\maketitle
\thispagestyle{empty}
\section*{Introducción}
    Las fracciones continuadas tienen su comienzo como...
\section*{Sobre el Teorema de Lagrange}
    La motivación es demostrar el teorema de Lagrange

    \textbf{Teorema de Lagrange} Sea $\alpha \in \mathbb{R}$. La fraccion continuada de $\alpha$ es periodica si y solo si existen $a, b, c \in\mathbb{Z}$ con $a\not=0$ tales que $a\alpha^2 + b\alpha + c = 0$.

    Empecemos suponiendo que $\alpha$ es una fracción continuada periódica, de lo cual $\alpha= [a_0,a_1,...,a_{j,\theta}]$, donde $\theta=\overline{[b_0,b_1,..,b_{n-1}]}$, por la ecuación (7.8) se sigue que
        \[
        \theta =\dfrac{\theta h_{n-1}+h_{n-2}}{\theta k_{n-1}+k_{n-2}}
        \]
    y esta es una ecuación cuádratica en $\theta$, de aquí $\theta$ es un número irracional cuadrático o bien un número irracional pero el valor de cualquier fracción continuada simple infinita es irracional(dem), ahora $\alpha$ puede escribirse como 
        \[
        \alpha = [a_0,a_1,..,a_j,\theta]=\dfrac{\theta m + m'}{\theta q + q'}
        \]
    donde $m'/q'$ y $m/q$ son los dos últimos convergentes para $[a_0,a_1,...,a_j]$, pero $\theta$ es de la forma $(a + \sqrt{b})/c$ y de aquí $\alpha$ es de forma semejante, debido a que como con $\theta$ puede excluirse la posibilidad de que $\alpha$ sea racional

    Ahora supongamos que $\alpha$ es irracional cuadrático, entonces es de la forma $\alpha=(a+\sqrt{b})/c$, con los enteros $a,b,c,d>0$, $c\not=0$, el entero $b$ no es un cuadrado perfecto puesto que $\alpha$ es irracional. Multiplequemos el numerador y el denominador por $|c|$ para obtener 
        \[
        \alpha = \dfrac{ac + \sqrt{bc^2}}{c^2} \textnormal{o bien } \alpha=\dfrac{-ac+\sqrt{bc^2}}{-x^2}
        \]
    de acuerdo con que $c$ sea positiva o negativa. Así puede escribirse $\alpha$ de la forma
        \[
        \alpha = \dfrac{m_0+\sqrt{d}}{q_0}
        \]
    donde $q_0|(d-m_0^2)$, $d,m_0$ y $q$ son enteros, $q_0\not=0$, $d$ no es un entero cuadrado perfecto. Escribiendo $\alpha$ en esta forma podemos obtener una formulación sencilla de su desarrlo fraccionario continuado $\overline{[a_0,a_1,...]}$ . Se probara que las siguientes ecuaciones 

        \[
        a_i=[\alpha_i], \qquad \alpha_i=\dfrac{m_i+\sqrt{d}}{q_i},
        \]
        \[
        m_{i+1}=a_iq_i-m_i, \qquad q_{i+1}=\dfrac{d-m_{i+1}^2}{q_i}
        \]
    definen las sucesiones infinitas de los enteros $m_i,q_i,a_i$ e irracionales $\alpha_i$ de tal forma que se cumplen las ecuaciones $(7.7)dem$ y de donde se obtendra el desarrollo continuado de $\alpha$

    En primer lugar, empecemos con 
\end{document}
