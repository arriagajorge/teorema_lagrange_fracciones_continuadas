\documentclass[11pt, article]{article}
\pagestyle{empty}
% pre\'ambulo

\usepackage{lmodern}
\usepackage[T1]{fontenc}
%\usepackage[spanish,activeacute]{babel}
\usepackage[spanish,es-tabla]{babel}
\usepackage{mathtools}
\usepackage{babel}
\usepackage{textcomp}
\usepackage[utf8]{inputenc}
\usepackage{vmargin}
\usepackage{amsfonts}
\usepackage{mathrsfs}
\usepackage{graphicx}
\usepackage{amsthm}
\usepackage{enumerate}
\usepackage{subcaption}
\usepackage{color, verbatim} % Paquetes que uso en mis comentarios.
%\usepackage[utf8]{inputenc}
\usepackage{amsthm} % Paquete para teoremas
\newtheorem{theorem}{Teorema} %Teorema
\newtheorem{definition}{Definición} 
\setpapersize{A4}
\setmargins{1.5cm}             % margen izquierdo
{1.5cm}                        % margen superior
{18cm}                         % anchura del texto
{25.42cm}                      % altura del texto
{10pt}                         % altura de los encabezados
{0cm}                          % espacio entre el texto y los encabezados
{0pt}                          % altura del pie de página
{1cm}                          % espacio entre el texto y el pie de página

\title{Fracciones continuadas}
\author{}
\date{}
\renewcommand{\baselinestretch}{1.5}

\begin{document}

    \begin{center}
    {\scshape\LARGE Análisis Matemático \par}

    \vspace{0.5cm}
    \large{\itshape{Manuel Ávila Guerrero}}  \\ 
    \large{\itshape{Jonathan Ramírez Ramírez}}  \\
    \large{\itshape{Jorge Vasquez Arriaga}} \\

    {\LARGE \textbf{Fracciones Continuadas} \par}
    \end{center}


\section*{Introducción}
    $1 + \sqrt{5}/2=\Phi$, es un número que cumple $\Phi^2=\Phi + 1$, si dividimos por $\Phi$ obtenemos $\Phi=1+1/\Phi$, si sustituimos, la $\Phi$ del lado derecho obtenemos,  $\Phi = 1 + \dfrac{1}{1 + 1 /\Phi}$

    si reiteramos el proceso 
    \[
    \Phi=1+\dfrac{1}{1+\dfrac{1}{1+\dots}}.
    \]
    
    Una expresión de la forma 
    \[
    a_0 + \dfrac{b_0}{a_1+ \dfrac{b_1}{a_2\dots}} 
    \]
    es llamada fracción continuada, en general $a_0, a_1,a_2,...$ y $b_0, b_1,b_2,...$ son numeros reales o complejos y el número de terminos puede ser tanto finito o infinito, si $1=b_0=b_1=b_2=...$ y cada $a_i$ es un entero positivo, para $i$ en los naturales, entonces diremos que es una fracción continuada simple.
    
    \textbf{Notación} Usaremos la expresión $[a_0,a_1,a_2, \dots]$ para designar a la fracción continuada simple.
    
\section*{Fracciones continuadas simples}
    
    \begin{definition}
    
    
    Sea $a_0,a_1,a_2...$ una sucesión infinita de enteros, todos positivos excepto tal vez $a_0$. Definimos inductivamente dos sucesiones de naturales $\{h_n\}_{n\geq 1}$ y $\{k_n\}_{n\geq 1}$ del modo siguiente,
        \[
        h_{-2}=0, \qquad h_{-1} =1, \qquad h_i=a_ih_{i-1} + h_{i-2} \qquad \textnormal{para } i\geq 0
        \]
        \[
        k_{-2}=1, \qquad k_{-1} =0, \qquad k_i=a_ik_{i-1} + k_{i-2} \qquad \textnormal{para } i\geq 0.
        \]
        \end{definition}
        \begin{theorem}
        Para cualquier número real positivo $x$,
        \[
        [a_0,a_1,...,a_{n-1}, x]=\dfrac{xh_{n-1}+h_{n-2}}{xk_{n-1}+k_{n-2}}. \label{teo1}
        \] 
        \end{theorem}
    \begin{proof} Procedamos por inducción. Si $n=0$
        \[
        x=\dfrac{xh_{-1}+h_{-2}}{xk_{-1}+k_{-2}}=\dfrac{x(1)+0}{x(0)+1}=x
        \]
        lo cual es verdadero. Ahora supongamos que el resultado se cumple para $[a_0,a_1,...,a_{n-1},x]$, entonces $[a_0,a_1,...,a_n,x]=[a_0,a_1,...,a_{n-1},a_n + 1/x]$ y por hipótesis de inducción
    \begin{align*}
        [a_0,a_1,...,a_{n-1},a_n + \dfrac{1}{x}]= 
            \dfrac{(a_n+1/x)h_{n-1}+h_{n-2}}{(a_n+1/x)k_{n-1}+k_{n-2}}= \\
            \dfrac{x(a_nh_{n-1}+h_{n-2})+h_{n-1}}{x(a_nk_{n-1}+k_{n-2})+k_{n-1}}= 
            \dfrac{xh_n+h_{n-1}}{xk_n+k_{n-1}}\\
    \end{align*}
    \end{proof}
    
    
    \begin{definition} Se define $r_n=[a_0,a_1,..,a_n]$ para toda $n$ número natural
    \end{definition}
    
    \begin{theorem}
        Para todos los enteros $n \geq 0$, se tiene $r_n=h_n/k_n$.
    \end{theorem} 

    
    \begin{proof}
        Aplicando el teorema \ref{teo1}, tenemos
    \[
    r_n=[a_0,a_1,...,a_n]=\dfrac{a_nh_{n-1}+h_{n-2}}{a_nk_{n-1}+k_{n-2}}=\dfrac{h_n}{k_n}\]
    \end{proof} 
    
    \begin{theorem}
    Los valores $r_n$, satisfacen las desigualdades, $r_0<r_2<r_4,...,r_3<r_1$, y $\lim_{n\to\infty}r_n$ existe 
    \end{theorem}
    \begin{proof}
        Primero probaremos que $h_ik_{i-1}-p_{i-1}q_{i}=(-1)^i$, con $i\geq 0$.
    Por inducción. Si $i=0$
        \[ 
        h_0k_{-1}-h{-1}k_0=1(1)-0(0)=1=(-1)^0
        \]
    se cumple el caso base, así que supongamos que se cumple para $i=m$, entonces para $m+1$
    \[
    h_{m+1}k_{m}-h_{m}k_{m+1}=(a_{m+1}h_{m} + h_{m-1})k_m-h_m(a_{m+1}k_m+k_{m-1})
    \]
    \[
    =a_{m+1}h_mk_m+h_{m-1}k_m-a_{m+1}h_mk_m-h_mk_{m-1}
    =(-1)(h_mk_{m-1}-h_{m-1}k_m)
    \]
    por hipótesis de inducción 
        \[
        (-1)(h_m k_{m-1}-h_{m-1}k_m)=(-1)(-1)^m=(-1)^{m+1}
        \]
    de la observación anterior, si dividimos por $k_nk_{n-1}$, obtenemos 
        \[
        \dfrac{h_n}{k_n}-\dfrac{h_{n-1}}{k_{n-1}}=\dfrac{(-1)^n}{k_nk_{n-1}},
        \]
    ahora calculemos $r_n-r_{n-2}$,
    
        \[
        r_n-r_{n-2}=\dfrac{h_n}{k_n}-\dfrac{h_{n-2}}{k_{n-2}}=\dfrac{h_nk_{n-2}-h_{n-2}k_n}{k_nk_{n-2}},
        \]
        notemos que 
        \[
        \dfrac{h_nk_{n-2}-h_{n-2}k_n}{k_nk_{n-2}}=(a_n h_{n-1}+h_{n-2})k_{n-2}-h_{n-2}(a_n k_{n-1}+k_{n-2})
        \]
        \[
        =a_n(h_{n-1}k_{n-2}-h_{n-2}k_{n-1})=a_n(-1)^{n-1},
        \]
    por lo que $r_n-r_{n-2}=a_n(-1)^{n-1}/q_nq_{n-1}$, además $r_n-r_{n-1}=(-1)^n/k_nk_{n-1}$, de lo cual si $n$ es par $r_n-r_{n-1}$ es mayor que cero y $r_n-r_{n-2}$ es menor que cero, en cambio si $n$ es impar $r_n-r_{n-1}$ es positivo y $r_n-r_{n-2}$ es no positivo, de lo cual obtenemos
    \[
    r_0<r_2<...<r_3<r_1.
    \]
    Ahora la sucesión $r_0,r_2,r_4,...$ definen una sucesión creciente monótona, y acotada superiormente por lo cual tiene un límite. Igualmente $r_1,r_3,r_5,..$ es creciente monótona y acotada inferiormente por lo que también tiene un límite, pero la diferencia entre $r_i-r_{i-1}$ tiende a cero conforme $i$ tiende hacia cero conforme $i$ tiende a infinito, ya que los enteros $k_i$ son crecientes con $i$, por lo cual el límite existe. 
    \end{proof}
    \begin{definition}
    Una sucesión infinita $a_0, a_1, a_2,...$ de enteros positivos exepto tal vez $a_0$, determina una fracción continuada simple infinita $[a_0, a_1, ..., ]$. El valor de $[a_0, a_1, ..., ]$ está definido como $\lim_{n\to\infty}[a_0, a_1, ..., a_n]$.
    \end{definition}  
    Otra forma de escribir este limite es $\lim_{n\to\infty} h_n/k_n$. El número racional $[a_0, a_1, ...,a_n]=h_n/k_n=r_n$ se llama el $n-$ésimo convergente.

    \begin{theorem}
        El valor de cualquier fracción continuada simple infinita $[a_0,a_1,a_2,...]$ es irracional.
    \end{theorem} 
    
    \begin{proof}
       Escribimos $\xi$ por $[a_0,a_1, a_2, ...]$, observamos que $\xi$ se encuentra entre $r_{n+1}$ y $r_n$, de lo que obtenemos $0<|\xi - r_n|<|r_{n+1}-r_n|$, vimos que $|r_{n+1}-r_n|=(-1)^n/(k_nk_{n-1})$, de lo cual multiplicando por $k_n$, obtenemos $0<|k_n\xi-h_n|<1/k_{n-1}$.
       
       Ahora supongamos que $\xi$ es racional, digamos $\xi=a/b$ con $a,b$ enteros positivos. Entonces si multiplicamos la desilgualdad anterior por $b$, obtenemos $0<|k_na-h_nb|<b/k_{n-1}$.
       Los enteros $k_n$ crecen conforme crece $n$, de manera que escogiendo $n$ lo suficientemente grande de manera que $b<k_{n+1}$. Entonces el entero $|k_na-h_nb|$ estaria entre $0$ y $1$, lo cual es imposible.
    \end{proof}
        
    Para desarrollar una fracción continuada $\xi$ simple infinita, definimos $a_0=\lfloor \xi_0 \rfloor$, $[\xi_1]=1/(\xi-a_0)$ y en seguida $a_1= \lfloor \xi_1 \rfloor, \xi_2=1/(\xi_1-a_1)$, y así mediante una función inductiva tenemos
        \begin{equation}
        a_i=\lfloor \xi_i \rfloor, \qquad \xi_{i+1}=\dfrac{1}{\xi_i-a_i}.
        \label{ecuacion_1}
        \end{equation}

    Por definición todos los $a_i$ son enteros, los $\xi_i$ son todos irracionales pues la irracionalidad de $\xi_1$ es consecuencia de $\xi$, la de $\xi_2$ es consecuencia de la de $x_2$, y así sucesivamente. 
    
    Ahora escribamos 
    \[ 
    \xi=[a_0,a_1,...,a_n.1,\xi_n]=\dfrac{\xi_nh_{n-1}+h_{n-2}}{\xi_nk_{n-1}+k_{n-2}}.
    \]
    Calculemos el valor de $\xi-r_{n-1}$
    \begin{align*}
        \xi-r_{n-1}=\xi-\dfrac{h_{n-1}}{k_{n-1}}=\dfrac{\xi_nh_{n-1}+h_{n-2}}{\xi_nk_{n-1}+k_{n-2}}-\dfrac{h_{n-1}}{k_{n-1}}\\
        =\dfrac{-(h_{n-1}k_{n-2}-h_{n-2}k_{n-1})}{k_{n-1}(\xi_nk_{n-1}+k_{n-2})}=\dfrac{(-1)^{n-1}}{k_{n-1}(\xi_nk_{n-1}+k_{n-2})}.
    \end{align*}
    Esta fracción tiende hacia cero conforme $n$ tiende a infinito, debido a que los enteros $k_n$ son crecientes con $n$ y $\xi_n$ es positivo. De aquí que $\xi_n-r_{n-1}$ tiende hacia cero cuando $n$ tiende al infinito y entonces
        \[
        \xi=\lim_{n\to\infty} r_n=\lim_{n\to\infty}[a_0,a_1,...,a_n]=[a_0,a_1,a_2,...]
        \]
        
    \subsection*{Fracciones continuadas periódicas}
    
    \begin{definition}
    Se dice que una fracción continuada simple infinita $[a_0,a_1,a_2,...]$ es periódica si existe un entero $n$ tal que $a_r=a_{n+r}$, para todos los enteros $r$ suficientemente grandes.
    \end{definition} 
    
    Por lo tanto una fracción continuada periódica $\xi$ puede escribirse de la siguiente forma
        \begin{align*}
         \xi=[a_0,a_1,a_2,...,a_j,b_0,b_1,...,b_{n-1},b_0,b_1,..,b_{n-1},b_{0},...] 
        =[a_0,a_1,a_2,...,a_j,\overline{b_0,b_1,...,b_{n-1}}].
        \end{align*}
        \
        Notemos que por el teorema 1 podemos escribir el valor de una fracción continuada simple $\xi$ como 

    \begin{equation}
    \xi=[a_0,a_1,...,a_{n-1},\xi_n]=\dfrac{\xi_n h_{n-1} + h_{n-2}}{\xi_n k_{n-1} + k_{n-2}}  \label{ecuacion_8}
    \end{equation}
    donde $\xi_n=\overline{[b_0,b_1,..,b_{n-1}]}$.
    

    \begin{theorem}[Teorema de Lagrange] 
        Sea $\alpha \in \mathbb{R}$. La fraccion continuada de $\alpha$ es periódica si y solo si existen $a, b, c \in\mathbb{Z}$ con $a\not=0$ tales que $a\alpha^2 + b\alpha + c = 0$.
    \end{theorem}

    Supongamos que $\alpha$ es una fracción continuada periódica, de lo cual $\alpha= [a_0,a_1,...,a_{j},\theta]$, donde $\theta=\overline{[b_0,b_1,..,b_{n-1}]}$, por el teorema $\ref{teo1}$ se sigue que
        \[
        \theta =\dfrac{\theta h_{n-1}+h_{n-2}}{\theta k_{n-1}+k_{n-2}}
        \]
    y esta es una ecuación cuadrática en $\theta$, de aquí $\theta$ es un número irracional cuadrático o bien un número irracional, pero el valor de cualquier fracción continuada simple infinita es irracional, de lo cual $\alpha$ puede escribirse como 
        \[
        \alpha = [a_0,a_1,..,a_j,\theta]=\dfrac{\theta m + m'}{\theta q + q'}
        \]
    donde $m'/q'$ y $m/q$ son los dos últimos convergentes para $[a_0,a_1,...,a_j]$, pero $\theta$ es de la forma $(a + \sqrt{b})/c$ y de aquí $\alpha$ es de forma similar, debido a que como con $\theta$ puede excluirse la posibilidad de que $\alpha$ sea racional.

    Ahora supongamos que $\alpha$ es irracional cuadrático, entonces es de la forma $\alpha=(a+\sqrt{b})/c$, con enteros $a,b,c>0$, $c\not=0$, el entero $b$ no es un cuadrado perfecto, pues  $\alpha$ es irracional. Multipliquemos el numerador y el denominador por $|c|$ para obtener 
        \[
        \alpha = \dfrac{ac + \sqrt{bc^2}}{c^2} \quad \textnormal{ o bien } \quad \alpha=\dfrac{-ac+\sqrt{bc^2}}{-c^2}
        \]
    de acuerdo con que $c$ sea positiva o negativa. $\alpha$ puede escribirse de la forma
        \[
        \alpha = \dfrac{m_0+\sqrt{d}}{q_0}
        \]
    donde $q_0|(d-m_0^2)$, $d,m_0$ y $q$ son enteros, $q_0\not=0$ y $d$ no es un entero cuadrado perfecto. Escribiendo $\alpha$ en esta forma podemos obtener una formulación sencilla de su desarrollo fraccionario continuado $\overline{[a_0,a_1,a_2,\dots]}$. Se probara que las siguientes ecuaciones 

        \begin{equation}
        a_i= \lfloor \alpha_i \rfloor, \qquad \alpha_i=\dfrac{m_i+\sqrt{d}}{q_i}, \label{prueba_ecuacion}
        \end{equation}
        \[
        m_{i+1}=a_iq_i-m_i, \qquad q_{i+1}=\dfrac{d-m_{i+1}^2}{q_i}
        \] 
    definen las sucesiones infinitas de los enteros $m_i,q_i,a_i$ e irracionales $\alpha_i$ de tal forma que se cumplen las ecuaciones $(\ref{ecuacion_1})$ y de donde se obtendra el desarrollo continuado de $\alpha$.

    En primer lugar, empecemos con $\alpha_0,m_0,q_0$ tal y como se determinaron anteriormente,  hagmaos $a_0=[\alpha_0]$. Si se conocen $\alpha_i,m_i,q_i,a_i$, entonces se tiene que 
        \[
         m_{i+1}=a_iq_i - m_i, q_i=(d-m_{i+q}^2)/q_i, \alpha_{i+1}=(m_{i+1} + \sqrt{d})/q_{i+1}, a_i=[\alpha_{i+1}].
         \]
    
    Es decir las ecuaciones anteriores (\ref{prueba_ecuacion}), realmente determinan las sucesiones $\alpha_i,m_i,q_i,a_i$ que son por lo menos reales.
    
    Ahora se aplica inducción para probar que los $m_i$ y los $q_i$ son enteros tales que $q_i\not=0$ y $q_i|(d-m^2)$. Esto se cumple para $i=0$. Supongamos que es cierto para el $i-$ésimo paso entonces, observamos que $m_{i+1}=a_iq_i-m_i$ es un entero. Entonces la ecuación 
         \[
        q_{i+1}=\dfrac{d-m_{i+1}^2}{q_i}=\dfrac{d-m_i^2}{q_i} + 2a_im_i-a_i^2q_i
        \]
    establece que $q_{i+1}$ es un entero. Es más $q_{i+1}$ no puede ser cero, dado que si lo fuera, se tendría $d=m^2_{i+1}$, pero $d$ no es un cuadrado perfecto. Finalmente, se tiene que $q_i=(d-m_{i+1}^2)/q_{i+1}$, de modo que $q_{i+1}|(d-m_{i+1}^2)$.
    
    Notemos que 
        \[
        \alpha_i-a_i=\dfrac{-a_iq_i+m_i+\sqrt{d}}{q_i}=\dfrac{\sqrt{d}-m_{i+1}}{q_i}=\dfrac{d-m_{i+1}^2}{q_i(\sqrt{d}+m_{i+1})}=\dfrac{q_i+1}{\sqrt{d}+m_{i+1}}=\dfrac{1}{\alpha_{i+1}}
        \]
    lo cual verifica $(\ref{ecuacion_1})$ y así se ha probado que $\alpha=[a_0,a_1,a_2,...]$, con los $a_i$ definidos por las ecuaciones ($\ref{prueba_ecuacion}$).
    
    Mediante $\alpha^{'}_{i}$ denotamos el conjunto de $\alpha_i$, esto es $\alpha^{'}_{i}=(m_i-\sqrt{d})/q_i$. Dado que el conjunto de un cociente es igual al cociente de lo conjugado se obtiene la ecuación
        \[
        \alpha_0{'}=\dfrac{\alpha_n^{'}h_{n-1}+h_{n-2}}{\alpha_n^{'}k_{n-1}+k_{n-2}}
        \]
    tomando los conjugados en $(\ref{ecuacion_8})$ y resolviendo para $\alpha_n^{'}$ se tiene
        \[
        \alpha_n^{'}=-\dfrac{k_{n-2}}{k_{n-1}}\left(\dfrac{\alpha_0^{'}-h_{n-2}/k_{n-2}}{\alpha_0^{'}-h_{n-1}/k_{n-1}}\right).
        \]
        
    Conforme $n$ tiende al infinito, tanto $h_{n-1}/k_{n-1}$ como $h_{m-2}/k_{n-2}$ tienden hacia $\alpha_0^{'}$, de lo cual la fracción que se encuentra dentro del paréntesis tiende a 1. Así que para un $n$ lo suficientemente grande, digamos $n>N$, 
    donde $N$ es fijo, la fracción del paréntesis es positiva y $\alpha_n^{'}$ es negativo, pero recordemos que $\alpha_n^{'}$ es positivo para toda
    $n\geq 1$ y de aqí que $\alpha_n - \alpha_n^{'}>0$ para $n>N$. Aplicando (\ref{prueba_ecuacion}), obtenemos $2 \sqrt{d}/q_{n} >0$
    de aquí que $q_n>0$ para $n>N$.
    
    De (\ref{prueba_ecuacion}), también se deduce que 
        \[
        q_nq_{n+1}=d-m^2_{n+1}\leq d, \qquad q_n\leq q_nq_{n+1}\leq d 
        \]
        \[
        m^2_{n+1} < m_{n+1}^2+q_n q_n+1=d, \qquad |m_{n+1}|<\sqrt{d}
        \]
    para $n>N$. Pues supusimos que $d$ es un entero positivo fijo, se concluye que $q_n$ y $m_{n+1}$, pueden saumir un solo numero fijo de valores posibes para $q_n$ y $m_{n+1}$ pueden asumir sólo un número fijo de valores posibles para $n>N$. De aquí que las parejas ordenadas $(m_n,q_n)$ pueden asumir sólo un número fijo de valores posibles, de la parejas para $n>N$, y por tanto existen enteros distints $j$ y $k$ tales que $m_j=m_k$ y $q_j=q_k$, pueden suponerse que se han escogido $j$ y $k$ tales que $j<k$. Por (\ref{prueba_ecuacion}), esto implica que $\alpha_j=\alpha_k$ y de aquí que
        \[
        \alpha=[a_0,a_1,...,a_{j-1},\overline{a_j,a_{j+1},...,a_{k-1}}].
        \]
    \section*{Aplicaciones de las Fracciones Continuadas}
     A continuación presentamos algunas de las aplicaciones de las fracciones continuadas, son la mejor aproximación posible a números irracionales, tambien la teoría de las fracciones continuadas son usadas para resolver ecuaciones diofánticas, llamdas de Pell, y se puede ver a través de los siguientes teoremas.
     
     \begin{theorem}
         Si $\xi$ es un número irracional, $a/b$ es un número racional, con denomidador positivo tal que $|\xi - a/b|<|\xi-h_n/k_n|$ para algún $n\geq 1$, entonces $b>k_n$.
     \end{theorem} 
     
    \begin{theorem}
        Sean  $a,b,c$ números enteros, $(h_n)_{n\geq 1}, (K_n)_{n\geq 1}$ las sucesiones anteriormente definidads, se tiene que la solución a, $ax+by=c$ donde $(a,b)=1$, es el par $x_0 = (-1)^{n-1} (h_{n-1})c,\quad y_0=(-1)^n (k_{n-1})c$.
     
    \end{theorem} 
    
     \section*{Glosario de simbología}
     
     $(a,b)$ : máximo comun divisor entre a y b.
     
     $\lfloor x \rfloor$ : parte entrera  de $x$ o función piso de $x$.
     
     $|x|$ : valor absoluto de $x$.

     \section*{Referencias bibliográficas}
     Niven I., Zuckerman S. H., Montgomery L. H., \textit{An introduction to the theory of numbers}, Wiley, 1991.
    
      Olds C. D., \textit{Continued fractions}, New mathematical library, Random House, 1963.
     
     Silverman H. Joseph, \textit{A friendly introduction to number theory}, Pearson, 2012.
     
\end{document}
